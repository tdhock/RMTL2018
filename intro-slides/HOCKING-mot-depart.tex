\documentclass{beamer}
\usepackage{tikz}
\usepackage[all]{xy}
\usepackage{amsmath,amssymb}
\usepackage{hyperref}
\usepackage{graphicx}
\usepackage{algorithmic}
\usepackage{textpos} 

\DeclareMathOperator*{\argmin}{arg\,min}
\DeclareMathOperator*{\Lik}{Lik}
\DeclareMathOperator*{\PoissonLoss}{PoissonLoss}
\DeclareMathOperator*{\Peaks}{Peaks}
\DeclareMathOperator*{\Segments}{Segments}
\DeclareMathOperator*{\argmax}{arg\,max}
\DeclareMathOperator*{\maximize}{maximize}
\DeclareMathOperator*{\minimize}{minimize}
\newcommand{\sign}{\operatorname{sign}}
\newcommand{\RR}{\mathbb R}
\newcommand{\ZZ}{\mathbb Z}
\newcommand{\NN}{\mathbb N}
\newcommand{\z}{$z = 2, 4, 3, 5, 1$} 

\newcommand{\algo}[1]{\textcolor{#1}{#1}}
\definecolor{PDPA}{HTML}{66C2A5}
\definecolor{CDPA}{HTML}{FC8D62}
\definecolor{GPDPA}{HTML}{4D4D4D}


\addtobeamertemplate{frametitle}{}{%
\begin{textblock*}{100mm}(.85\textwidth,-0.8cm)
\includegraphics[height=1cm,width=2cm]{Logo}
\end{textblock*}}

\begin{document}

% \begin{frame}
%   \frametitle{Practical}
% Please no food nor drinks in SH2800


% %1) Les participants peuvent interpeller les organisateurs reconnaissables à leur tags spéciaux (surprise !)

% %2) Perhaps show a map of the science complex to point the location of the cocktail ?
% %1920x1080
% \end{frame}

\begin{frame}
  \frametitle{Commanditaires / Sponsors}
  \includegraphics[width=\textheight]{sponsors}
\end{frame}

\begin{frame}
  \frametitle{L'ann\'ee prochaine / Next year}
  \includegraphics[width=\textwidth]{logoRAQuébec2019}
\end{frame}


\end{document}

